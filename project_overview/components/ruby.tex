\subsection{Ruby}
\paragraph*{Language features}
Ruby\cite{rubylang} is a interpreted, dynamic language, embracing a variety of programming paradigms. It contains many features designed to increase its extensibility via meta-programming.

Execution centres around the creation of objects, every class inherits from at least \verb|BasicObject|. Function invocation is triggered by sending the target object a \emph{message}.

Upon receipt of a message, it is handled by the lowest level of an objects inheritance-hierarchy \textemdash{} the composed chain of class and module extensions that are mixed-into the object instance. A level will either respond to the message or, if unable to, propagate the request further up the chain.

A common meta-programming technique is to redefine how a module within the hierarchy responds when a method definition is missing. Instead of simply passing the message to the next level, it can inspect the name and arguments of the function, or its own state, and give a response. This cancels further progression of the request.

\begin{comment}
\begin{lstlisting}[
  language=Ruby,
  label=lst:methmiss,
  caption=A toy example where an object responds to a missing method instead of propagating the message.
]
class HungryHippo
  def eat
    puts "Nom nom nom!"
  end

  def method_missing(meth, *args, &block)
    if /eat/.match meth.to_s
      puts "Eat? ok then."
      eat
    else
      super
    end
  end
end

hippo = HungryHippo.new
hippo.eat
hippo.dont_eat
# >> Nom nom nom!
# >> Eat? ok then.
# >> Nom nom nom!
\end{lstlisting}
\end{comment}

This flexibility is often used (or abused) to reduce the amount of boilerplate code written.

In addition to dynamically responding to received methods, objects are often substituted for objects of another class that have subtly different behavior. This will be successful as long as they respond the same method calls. Therefore, it is common practice to consider only the interface presented by interacting objects and not their individual types.

The benefit of this \emph{duck-testing} is that the same series of of method requests, present in a line of code, can cause very different patterns of computation. If the response of a single link in the chain is altered, the programmer would be unaware, and unconcerned, as long as the expected pipeline result is returned.

This technique of redirecting computation will be explored by the RubiCL library. It allows a decoupling of the programmer's requests and the underlying, massively-parallel implementation.

\paragraph*{Native extensions}
The latest versions of the Ruby language make it simple to produce native extensions. Functionality is provided by C shared-objects that interact with the underlying \ac{RubyVM}.

\begin{lstlisting}[
  language=C,
  label=lst:native_ext1,
  caption=The C code defining a module with the ability to perform native actions.
]
#include "ruby.h"
#include "something_native.h"

/* All Ruby objects are of type VALUE and must be unboxed */
/* Every method takes self as an explicit argument */
VALUE
methodSomethingNative(VALUE self, VALUE int_param_object) {
    int param  = FIX2INT(int_param_object);
    int result = doSomethingNative(param);

    return INT2FIX(result);
}

void /* module_example is name defined in Makefile */
Init_module_example() {
    VALUE ModuleEx = rb_define_module("ModuleExample");
    /* Visibility, Module, Name, Method, Arg count. */
    rb_define_private_method(
        ModuleEx, "something_native",
        methodSomethingNative, 1);
}
\end{lstlisting}

Adding native methods is as simple as performing the heavy-lifting as one would in a pure C application. The \verb|ruby.h| library then allows the programmer to create a Ruby module or class with mappings between method names and the underlying implementations.

\begin{lstlisting}[
  language=Ruby,
  label=lst:native_ext2,
  caption=A Ruby class utilising a native extension module.
]
require './module_example'

class NativeThing
  include ModuleExample

  def method_requiring_native
    something_native(1)
  end
end
\end{lstlisting}

The required complication of converting between Ruby objects and basic C types is handled via macros, defined for all sensible conversions.

Once an extension has been compiled, the shared-object file is \verb|require|d and the constructed object is available, no differently to a pure Ruby implementation. In the case of RubiCL, modules for particular concerns are provided and then mixed-into classes that require native functionality.

\paragraph*{Suitability as the project's target language}
The project decided to use Ruby as the target language as, alongside Python, it is often recommended to beginners for analytics and \emph{data-science}. This is perhaps due to the syntax being relatively straightforward and often self-documenting.

Unlike Python, Ruby's design is less opinionated about the \emph{principle of least surprise}, and therefore makes it much easier to drastically extend its capability while hiding complexity from unaware users. 

In addition, the need for constant method-hierarchy lookup has been blamed for its poor performance. The potential for dynamic redefinition of methods complicates caching and can heavily impact certain compute-intensive tasks. This makes Ruby a suitable target for a project aiming to offer an optimised library for such tasks.
