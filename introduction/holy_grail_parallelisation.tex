\subsection{The holy grail of automatic parallelisation}
Improvements in language design and compilation are areas of study hoping to increase the magnitude of parallel execution in the wild, without requiring user interaction.

Researchers are investigating the feasibility of discovering parallelism, inherent in user code, through analysis \cite{auto_par}.

Whilst some breakthroughs have been made, progress is slow due to the massive complexity of the task.
Languages with user-managed memory are hard to perform dependency analysis on correctly. In addition, the seemingly limitless variety of user programs greatly complicates any blanket solution.

It is likely that automatic parallelisation compilers will not become useful to a developer  in the near future.

However, automatic parallelisation of code can be achieved if the scope of attempted transformation is reduced. In certain software paradigms, there is low-hanging fruit that can feasibly be parallelised automatically. This provides a stop-gap solution whilst research continues.
