\chapter{Results}

\section{Benchmarks}
\newpage
\subsection{Map tasks}
\begin{figure}[H]
  \centering
  \includegraphics[width=\textwidth]{./graphing/just_map/runtimes.pdf}
  \caption{Task duration by execution target for \emph{Map}.}
  \label{fig:map_task_runtime_g}

  \includegraphics[width=\textwidth]{./graphing/just_map/per_element.pdf}
  \caption{Task duration per processed element for \emph{Map}.}
  \label{fig:map_task_per_el_g}

\end{figure}
\begin{figure}[H]
  \centering
  \includegraphics[width=\textwidth]{./graphing/just_map/prop_van.pdf}
  \caption{Proportion of vanilla Ruby performance achieved for \emph{Map}.}
  \label{fig:map_task_vperf_g}

  \includegraphics[width=\textwidth]{./graphing/just_map/prop_bes.pdf}
  \caption{Proportion of bespoke C extension performance achieved for \emph{Map}.}
  \label{fig:map_task_bperf_g}
\end{figure}
\subsubsection{Observations}
Figure~\ref{fig:map_task_runtime_g} shows that all trialled alternatives exhibit similar performance when executing \emph{Map} tasks.
This is likely due to the computational simplicity of the task. The probable cause for the bottleneck is the need to move data in and out of the RubyVM's internal array structures.

The \ac{CPU} compute-devices installed in both machines outperform the \ac{GPGPU} over the range of tested datasets. This is easiest to observe in Figure~\ref{fig:map_task_per_el_g}.

When parallelising purely-map tasks, the project library performs favourably compared to the standard task implementation of Ruby $2.2$. Figure~\ref{fig:map_task_vperf_g} demonstrates that that, at best, a factor $3.5$\textendash$4$ speed-up over typical execution can be achieved on both systems.

The figure also shows that for all tested datasets, the smallest of which contained $1$ million elements, outsourcing computation to the library was beneficial.

In Figure~\ref{fig:map_task_bperf_g}, it can be seen that, on both systems, RubiCL throughput is not far from bespoke sequential code. On the laptop, the parallel \ac{CPU} implementation exceeds the non-parallel native extension. It achieves $1.02$ times the rate of processing on $19$ million elements.

However, the laptop \ac{CPU} presents $4$ hardware threads to \ac{OpenCL}, the native extension utilises only a single thread of execution. As they are performing roughly the same operation, it appears that here \ac{OpenCL}'s raw throughput increase is insignificant after processing-model overhead is accounted for.


\subsubsection{Analysis}
It is clear that the throughput with which Ruby is capable of performing \emph{Map} tasks has been significantly increased.
On the other hand, the project's parallel library does not significantly outperform a custom, tailored, sequential solution.
Yet, for all systems choosing the optimal device, no less than $80\%$ of the best-presented throughput is achieved at $10$ million elements, improving to no less than $95\%$ at $19$ million.
With the library providing automatic translation of all functions stated into near-optimal execution patterns, this deficit is insignificant. Much more significant is the mitigation of the need to write and compile native extensions for every required calculation.

\subsection{Dense Filter tasks}
\begin{figure}[H]
  \centering
  \includegraphics[width=\textwidth]{./graphing/dense_filter/runtimes.pdf}
  \caption{Task duration by execution target for dense \emph{Filter}.}
  \label{fig:dfilter_task_runtime_g}

  \includegraphics[width=\textwidth]{./graphing/dense_filter/per_element.pdf}
  \caption{Task duration per processed element for dense \emph{Filter}.}
  \label{fig:dfilter_task_per_el_g}

\end{figure}

\begin{figure}[H]
  \centering
  \includegraphics[width=\textwidth]{./graphing/dense_filter/prop_van.pdf}
  \caption{Proportion of vanilla Ruby performance achieved for dense \emph{Filter}.}
  \label{fig:dfilter_task_vperf_g}

  \includegraphics[width=\textwidth]{./graphing/dense_filter/prop_bes.pdf}
  \caption{Proportion of bespoke C extension performance achieved for dense \emph{Filter}.}
  \label{fig:dfilter_task_bperf_g}
\end{figure}

\subsubsection{Observations}
Figure~\ref{fig:dfilter_task_runtime_g} is less cluttered than the corresponding graph for \emph{Map} tasks, as there is more variation between the performance of dense \emph{Filter} implementations.
The figure suggests that \emph{Filter} tasks scheduled on the desktop system suffer from a higher latency than their counterparts on the laptop. This is signified by the comparatively elevated runtime for smaller datasets.

Further unlike \emph{Map} tasks, on both systems the \ac{GPGPU} compute-device outperforms the \ac{CPU} when filtering large datasets. For the laptop system, the domination is present on every tested dataset size. On the contrary, the desktop \ac{CPU} is initially a shade faster but is quickly dwarfed by the significantly higher throughput of the \ac{GPGPU} device.

On the laptop system, Figure~\ref{fig:dfilter_task_vperf_g} demonstrates that the library is beneficial when processing all datasets within the tested range. The desktop is not quite as successful at performing dense purely-\emph{Filter} tasks, hampered at smaller datasets by its increased latency. At least $5$ million elements are required before it is worthwhile to offload computation onto the \ac{GPGPU} via RubiCL. The \ac{CPU} trails not long after, at $8$ million elements, but never achieves more than $25\%$ speedup over the standard Ruby implementation.

On both test systems, when using the \ac{GPGPU}, noticeable speedup of \emph{Filter} operations can be achieved. With a $2.5$ times speedup on the desktop and over $3.5$ times on the laptop, large filtering operations are significantly accelerated by the RubiCL library.

Figure~\ref{fig:dfilter_task_bperf_g} shows that the laptop \ac{GPGPU} achieves a high proportion, peaking at $75\%$, of bespoke sequential code performance. The desktop system boasts a lesser proportion, at $50\%$, but the lack of plateau in the figure suggests that this gap would close further as datasets increase in size.

\subsubsection{Analysis}
The lacklustre performance for \ac{CPU} devices can be partly explained by the fact that the parallel implementation of filtering, although asymptotically identical, requires much more work than a sequential filter. In this case, the hidden constants involved for distributing computation have a large effect on the task duration, larger than that of parallelising the predicate scan along the data.
Nonetheless, the need to write and compile native extensions for each distinct query performed is again removed when using the RubiCL library. A system-dependent performance hit of between $25\%$ and $50\%$ behind a handwritten extension is far more justifiable when it facilitates rapid-prototyping, especially when compared to the $80\%$ penalty that Figure~\ref{fig:dfilter_task_bperf_g} highlights for unoptimised Ruby $2.2$.

\subsection{Sparse Filter tasks}
\begin{figure}[H]
  \centering
  \includegraphics[width=\textwidth]{./graphing/sparse_filter/runtimes.pdf}
  \caption{Task duration by execution target for sparse \emph{Filter}.}
  \label{fig:sfilter_task_runtime_g}

  \includegraphics[width=\textwidth]{./graphing/sparse_filter/per_element.pdf}
  \caption{Task duration per processed element for sparse \emph{Filter}.}
  \label{fig:sfilter_task_per_el_g}

\end{figure}

\begin{figure}[H]
  \centering
  \includegraphics[width=\textwidth]{./graphing/sparse_filter/prop_van.pdf}
  \caption{Proportion of vanilla Ruby performance achieved for sparse \emph{Filter}.}
  \label{fig:sfilter_task_vperf_g}

  \includegraphics[width=\textwidth]{./graphing/sparse_filter/prop_bes.pdf}
  \caption{Proportion of bespoke C extension performance achieved for sparse \emph{Filter}.}
  \label{fig:sfilter_task_bperf_g}
\end{figure}

\subsubsection{Observations and analysis}
Figure~\ref{fig:sfilter_task_runtime_g} looks very similar to that of dense filtering. Indeed, many of the observations are the same.

As before, the difference in latency between the laptop and desktop systems causes a significant proportional gap for smaller datasets.
It leaves sparse \emph{Filter} needing the same minimum dataset sizes for beneficial inclusion as dense \emph{Filter}.

Yet again, \ac{GPGPU} devices are dominant among the \ac{OpenCL} filtering implementations, for all but the smallest dataset on the desktop.

The bespoke extension performs much better comparatively at sparse filtering than dense filtering. Figure~\ref{fig:sfilter_task_vperf_g} shows a $9$ times performance speedup over unoptimised code, nearly twice the speedup of dense filtering.
With this in mind, Figure~\ref{fig:sfilter_task_bperf_g} shows a decrease in performance of RubiCL relative to bespoke filtering, compared to the denser predicate examined earlier.

The inverse is true in Figure~\ref{fig:sfilter_task_vperf_g}, with RubiCL demonstrating an improved $3$\textendash$5$ times speedup when executing on \ac{GPGPU} devices over Ruby $2.2$ for dense filtering.

The differences in relative performance between dense and sparse \emph{Filter} task implementations suggest that the cause for diverging ratios may result from there being fewer conditional insertions to the result vector. Alternatively, when less data is returned from a compute-device, the one-off latency penalty is more significant and may offset the benefits of relatively-high transfer bandwidth.

\subsection{MapFilter tasks}
\begin{figure}[H]
  \centering
  \includegraphics[width=\textwidth]{./graphing/mapfilter/runtimes.pdf}
  \caption{Task duration by execution target for \emph{MapFilter}.}
  \label{fig:mfilter_task_runtime_g}

  \includegraphics[width=\textwidth]{./graphing/mapfilter/per_element.pdf}
  \caption{Task duration per processed element for \emph{MapFilter}.}
  \label{fig:mfilter_task_per_el_g}

\end{figure}

\begin{figure}[H]
  \centering
  \includegraphics[width=\textwidth]{./graphing/mapfilter/prop_van.pdf}
  \caption{Proportion of vanilla Ruby performance achieved for \emph{MapFilter}.}
  \label{fig:mfilter_task_vperf_g}

  \includegraphics[width=\textwidth]{./graphing/mapfilter/prop_bes.pdf}
  \caption{Proportion of bespoke C extension performance achieved for \emph{MapFilter}.}
  \label{fig:mfilter_task_bperf_g}
\end{figure}
