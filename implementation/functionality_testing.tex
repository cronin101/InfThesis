\section{Functionality Testing}
A performant system for calculation parallelisation isn't much use if its behaviour is incorrect. In order to increase confidence that the system performs as expected, the library was developed alongside a comprehensive test suite.

Having significant tests around behaviour enables more significant alterations to occur smoothly. This allowed the pace of experimentation to increase. New ideas can be verified as enhancing performance without introducing behaviour regressions.

The \verb|RSpec|\cite{rspec} testing library was used to produce test-cases. It presents a \ac{DSL} for defining the intended behaviour of objects.

By describing a context corresponding to each feature that an subcomponent is designed to present, and testing boundary cases within that context, a rigid specification of correct behaviour was defined.

The advantages of testing were significant in terms of effort-economy. With a full test-suite execution taking less than $100$ms on the development laptop, it was responsive enough to be triggered by each updated file within the development directory. This immediately highlighted interface clashes and regressions introduced during development. In addition, it reduced the amount of time wasted, manually checking that the system performed as advertised.

Compared to the stressful development practices of other projects witnessed, this development style is subjectively judged to be a significant success of the project.

\todo{RSpec --format documentation in appendix}
