\section{Management System}

\subsection{Function parser}
The system's function parser is responsible for converting a supplied anonymous function into C syntax. The functionality of the parser is demonstrated in Listing~\ref{lst:parser_ex}
\begin{lstlisting}[
  language=Ruby,
  label=lst:parser_ex,
  caption=The \emph{LambdaBytecodeParser} converts an anonymous function Ruby object into an array of C expressions.
]
foo = 3
a_function = ->(x){ foo * (2 + x) }
#=> #<Proc:0x007f976207ff48@(pry):12 (lambda)>

parser = RubiCL::LambdaBytecodeParser.new(a_function)
#=> #<struct RubiCL::LambdaBytecodeParser
#   function=#<Proc:0x007f9761c362c0@(pry):15 (lambda)>>

parser.bytecode
#=> " == disasm: <RubyVM::InstructionSequence:block in __pry__
# == catch table
# | catch type: redo   st: 0000 ed: 0016 sp: 0000 cont: 0000
# | catch type: next   st: 0000 ed: 0016 sp: 0000 cont: 0016
# |-----------------------------------------------------------
# local table (size: 2, argc: 1 [opts: 0, rest: -1, post: 0,
#                                  block: -1, keyword: 0@3] s3)
#   [ 2] x<Arg>     
#   0000 trace            256                            (  22)
#   0002 trace            1
#   0004 getlocal         foo, 2
#   0007 trace            1
#   0009 putobject        2
#   0011 getlocal_OP__WC__0 2
#   0013 opt_plus         <callinfo!mid:+, argc:1, ARGS_SKIP>
#   0015 opt_mult         <callinfo!mid:*, argc:1, ARGS_SKIP>
#   0017 trace            512
#   0019 leave"
parser.parsed_operations
#=> [3, 2, "x", "+", "*"]

parser.to_infix
#=> ["3 * (2 + x)"]
\end{lstlisting}

The conversion process occurs over three stages: dumping bytecode, lexing, and reconstruction.

\paragraph*{Obtaining function bytecode}
The bytecode instructions produced by a compiled anonymous function object is provided by the \verb|RubyVM::InstructionSequence| module's \verb|disassemble| method.
It returns a human readable string that includes all stack-machine instructions.

\paragraph*{Lexing bytecode string}
Instructions of interest are extracted from the human-readable string. This is achieved via a regular expression containing a whitelist of keywords:
\begin{verbatim}
/(?:\d*\s*(?:(getlocal.*|putobject.*|opt_.*).?))/
\end{verbatim}

The instructions are then tokenised, by the process detailed in Listing~\ref{lst:tokeniser_rules}.
The end result is a list of tokens representing stack-machine instructions, in \ac{RPN}.

\begin{lstlisting}[
  language=Ruby,
  label=lst:tokeniser_rules,
  caption=Tokenisation rules for lexing human-readable bytecode.
]
def translate(operation)
  case operation
  # First function argument
  when /getlocal_OP__WC__0 #{function.arity + 1}/
    'x'

  # Second function argument
  when /getlocal_OP__WC__0 #{function.arity}/
    'y'

  # Indexed bound variable
  when /getlocal_OP__WC__1 \d+/
    id = /WC__1 (?<i>\d+)/.match(operation)[:i].to_i
    index = locals_table.length - (id - 1)
    beta_reduction locals_table[index]

  # Named bound variable
  when /getlocal\s+\w+,\s\d+/
    name = /getlocal\s+(?<name>\w+),/.match(operation)[:name].to_sym
    beta_reduction name

  # Literal Zero
  when /putobject_OP_INT2FIX_O_0_C_/
    0

  # Literal One
  when /putobject_OP_INT2FIX_O_1_C_/
    1

  # Floating-Point Literal
  when /putobject\s+-?\d+\.\d+/
    operation.split(' ').last.to_f

  # Integer Literal
  when /putobject\s+-?\d+/
    operation.split(' ').last.to_i

  # Method Sending
  when /opt_send_simple/
    /mid:(?<method>.*?),/.match(operation)[:method].to_sym

  # Built-in Operator
  when /opt_/
    LOOKUP_TABLE.fetch operation[/opt_\w+/].to_sym
  else
    raise "Could not parse: #{operation} in #{bytecode}"
  end
end

def beta_reduction variable_name
  function.binding.local_variable_get variable_name
end

\end{lstlisting}

\paragraph*{Expression reconstruction}
The final stage of the conversion pipeline. It requires converting \ac{RPN} to infix form.
