\subsection{Interface}
The produced system is able to interact seamlessly with existing Ruby code, via type annotation on collection objects.

\begin{lstlisting}[
  language=Ruby,
  label=lst:example_snippet,
  caption=Redirecting a computation through the RubiCL library via type annotation.
]
  # Sequential stdlib code
  (1..1_000_000)
    .map { |x| x + 15 }
    .select { |y| y % 15 == 0 }

  # Parallel code using RubiCL
  (1..1_000_000)[Int]
      .map { |x| x + 15 }
      .select { |y| y % 15 == 0 }[Fixnum]
\end{lstlisting}


When a user is sure that all objects within an \verb|Enumerable| are of a single, basic type, they can append a type declaration to the container. This declaration lies within the method pipeline and states the equivalent C type, the object is then wrapped by the RubiCL execution environment.

Further method calls are captured by the \verb|Device| instance handling the dataset, and pushed onto a work-queue.

Eventually, a result is requested, either by a user casting back to a Ruby object class, or by performing a terminal action such as summation. The work-queue is then optimised and mapped to OpenCL kernels, dispatched to the target compute device.

The produced wrapper solution for including additional functionality to the Ruby runtime is ideal. Programmers must only grasp the concept of annotating type-conversion at the beginning and end of any calculation pipelines. All other syntax of the library is identical to normally-written Ruby code.

Despite the simplicity of the library's presented interface, there is a lot of work going on behind the scenes. The technical details of which will be discussed in the \emph{Implementation} chapter.

As an overview, the steps undertaken by the RubiCL library for the example given in Figure~\ref{lst:example_snippet} include:
\begin{itemize}
  \item Moving the dataset elements into continuous memory addressable by the compute device.
  \item Recording the loaded dataset type, to allow static type-system operations.
  \item Parsing the \verb|block| argument of the \verb|#map| task's bytecode and constructing an equivalent C99 expression.
  \item Parsing the \verb|block| argument of the \verb|#select| task's bytecode and constructing an equivalent C99 expression.
  \item Inserting a \verb|Map| task at the beginning of the \verb|TaskQueue| to convert from Ruby objects to C \verb|int|s.
  \item Inserting a \verb|Map| task at the end of the \verb|TaskQueue| to convert from C \verb|int|s back to Ruby objects.
  \item Simplifying the $4$ tasks in the \verb|TaskQueue| to a single, \verb|MapFilter| task via \emph{fusion}.
  \item Generating the OpenCL kernel required to perform the \verb|MapFilter| task.
  \item Executing the produced kernel on the compute device, recording metrics.
  \item Releasing resources required by the OpenCL library during the task.
  \item Returning the resultant values as a Ruby array.
\end{itemize}

\subsection{Software architecture}
The library is constructed from the following set of modules and classes, alongside their responsibilities:
\begin{description}
  \item[RubiCL] Environment singleton and top level namespace.
The library's functionality is included in an application by \verb|require|ing this module. It handles the import of all sub-components of the runtime.
Other responsibilities include storing versioning metadata and selecting which available device should be the default compute target.

\paragraph*{Interface:}
\begin{description}
  \item[self.opencl\_device] Returns the current compute device. (Default: \verb|RubiCL::CPU|)
  \item[self.opencl\_device=(Device)] Sets the current compute device.
\end{description}

  \item[CastAccess] A module designed to extend container types with the ability to start a computation pipeline.
For example, the \verb|Array| built-in class is modified with \verb|Array.class_eval { include RubiCL::CastAccess }|.
This allows parallel primitives performed on \verb|Arrays| to be executed by the compute device following an annotation, such as \verb|[1, 2, 3][Int]|.
Upon casting, the actual conversion operations performed are specified by the target class. This module is decoupled from implementation and provides purely syntactic enhancements.

Traditionally in Ruby, invoking the \verb|[]| method of an \verb|Enumerable| is only used for indexed access to members. The standard implementation supports receiving integer arguments and returns the element at the given offset. It also supports \verb|Range| arguments and returns the corresponding continuous subset.

The decision was made that that massing a \verb|Class| constant here is something that would never occur in common use. Therefore, the RubiCL library uses occurrences of this calling behaviour to indicate that a dataset should be wrapped.

\paragraph*{Interface:}
\begin{description}
  \item[$\lbrack\rbrack$(Type)] Overridden on extended object to call the method provided by a \verb|Class| argument's \verb|rubicl_conversion| method on the current compute device, also providing the dataset it was called on. Behaviour when called with a non-\verb|Class| argument is unchanged.
\end{description}

  \item[Target C-type classes]
An observant reader may notice that the constant \verb|Int| is passed in Figure~\ref{lst:example_snippet} when signalling that the container should be transformed into C type \verb|int|s. This class is not defined within the standard library, instead \verb|Fixnum| is the container for fixed-precision integers that can be encoded within a single machine word.

The \verb|Int| class was constructed to represent the abstract type of C integers. In addition, the \verb|Double| type has been defined for floating-point numbers.

Each C-type class defines how to transfer an input dataset to the compute device, via methods defined by the \verb|BufferManager|.

\paragraph*{Interface:}
\begin{description}
  \item[self.rubicl\_conversion] Provides the method and type arguments to call on the current compute device, alongside a dataset, in order to load it.
\end{description}

\item[Native Ruby result classes]
At the end of the computation pipeline, results are retrieved either by casting back to a Ruby type, or by performing a reduction action such as summation.

The Ruby classes used to convert back to the calculation's result type are provided with the standard Ruby implementation: \verb|Fixnum| and \verb|Float|.

Mirroring the responsibilities of the C-type classes, additional static methods have been added to these classes to instruct the \verb|BufferManager| how to return a result dataset for the given type.
\paragraph*{Interface:}
\begin{description}
  \item[self.rubicl\_conversion] Provides the method to call on the current compute device, in order to retrieve the typed dataset.
\end{description}

  \item[BufferManager]
In order to prevent the \verb|Device| class becoming a \emph{god object}, manipulating the device buffer is performed by a service object. The \verb|BufferManager| provides an interface to load objects, specifying their C-type, and later retrieve them. The type of the currently loaded buffer is then stored, to assist kernel generation for queued parallel tasks.

The manager also provides caching of the dataset to prevent unnecessary retrieval if no operations have been performed.

The ability to interact with an \ac{OpenCL} buffer is provided by the \verb|BufferBackend| native extension module.
\paragraph*{Interface:}
\begin{description}
  \item[load(type: Type, object: Object)] Makes the provided object addressable by the \ac{OpenCL} compute device.

  \item[retrieve(type: Type)] Retrieves the resultant object from the compute device address space.

  \item[access(type: Type)] Returns a handle to the device address space, passed by \verb|Device| when executing tasks.
\end{description}

\item[Device]
  An abstract superclass, providing all functionality of the execution context during method pipelines. Instantiated as a singleton, in either \ac{GPU} or \ac{CPU} flavour. The subclass overrides only the initialisation procedure, passing the correct device-type flags to the \ac{OpenCL} \ac{API}, and provides a means to later differentiate between device types. Knowing which kind of device a kernel will execute on allows specific optimisations, such as avoiding \emph{bank conflicts} for \verb|Scan| tasks occurring on a \verb|GPU|.
\paragraph*{Interface:}
Where possible, all methods return the device context to allow method chaining.
\begin{description}
  \item[$\lbrack\rbrack$(Type)] Used to signal the end of a computation pipeline. Sends the method provided by \verb|Type.rubicl_conversion| to itself.

  \item[load\_object(Type, Object)] Delegates to the buffer manager.

  \item[sort] Enqueues a task to sort the buffer.

  \item[zip(Enumerable)] Flushes the current pipeline and then creates a tuple buffer from the result and the inputted \verb|Enumerable|.

  \item[fsts] Bifurcates a loaded tuple buffer, keeping only the first elements.

  \item[snds] Like the previous method, but keeps only the second elements.

\end{description}

\end{description}


\paragraph*{Example interaction} Figure~\ref{fig:rubicl_components} shows the interactions between classes during a typical parallelised computation.
\begin{sidewaysfigure}[h]
  \includegraphics[width=\textwidth]{./figures/arch_components.pdf}
  \caption{An overview of the interacting software components during the lifetime of a typical computation}
  \label{fig:rubicl_components}
\end{sidewaysfigure}

\subsection{Interacting with hardware devices}
Interaction with hardware devices present on the system occurs via native extensions. These extension modules are mixed-into device singletons, created when the library is first launched. Figure~\ref{fig:rubicl_devices} shows the functionality of these singletons and their subcomponents.

\begin{figure}[h]
  \includegraphics[width=0.8\textwidth]{./figures/arch_diagram.pdf}
  \caption{The RubiCL runtime maintains singletons for each device, used to trigger management functions and execute kernels.}
  \label{fig:rubicl_devices}
\end{figure}

Both \verb|CPU| and \verb|GPU| objects, tasked with managing device state, inherit from a common \verb|Device| superclass. The main difference in their implementation is differing initialisation procedure. Having two device types allows target-specific optimisation by the code generator, shown later.

The \verb|Device| subclasses delegate maintaining the list of tasks to a \verb|TaskQueue| object. In addition, they lack the ability to call memory management functions on devices and instead trigger functionality via an instance of \verb|DeviceService::BufferManager|.

Implementing all device logic that does not require hardware interoperability in Ruby made the system much easier to test. The time taken for the device control flow to execute is insignificant compared to the time taken for data processing. Writing this section in C would have been misguided as the performance benefits would not be worth the impaired rate of development.
